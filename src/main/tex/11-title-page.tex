\begin{center}
РЕФЕРАТ
\end{center}

% \large Московский авиационный институт\\[5.5cm]

% \huge Реферат \\[0.6cm] % название работы, затем отступ 0,6см
% \large на тему:  <<Метод идентификации музыкальных
% произведений по аудио фрагментам концертных исполнений>>\\[3.7cm]


% \end{center}

% \begin{flushright}
% Выполнил: студент гр. М8О-406Б \\
% Давид Гринберг \\
% \end{flushright}


% \vfill

% \begin{center}
% \large Москва 2020
% \end{center}

% \thispagestyle{empty}
Выпускная квалификационная работа содержит 43 страницы, \linebreak 15 рисунков, 3 таблицы, 7 листингов, 13 использованных источников.


КВАНТИЗАЦИЯ, МАШИННОЕ ОБУЧЕНИЕ, НЕЙРОННЫЕ СЕТИ, ОПТИМИЗАЦИЯ РАБОТЫ НЕЙРОСЕТЕЙ, C++, ЭФФЕКТИВНОЕ ПЕРЕМНОЖЕНИЕ МАТРИЦ, НИЗКОБИТОВЫЕ ВЫЧИСЛЕНИЯ, ОГРАНИЧЕННОСТЬ РЕСУРСОВ, ПРОМЕЖУТОЧНОЕ ГРАФОВОЕ ПРЕДСТАВЛЕНИЕ.

В данной работе реализуется квантизатор для перевода нейронных сетей в сжатый бинарный формат, в котором работа квантизованных слоёв реализовывается при помощи эффективного алгоритма перемножения матриц, основанного на работе побитовых операций.

В теоретической части работы приводится описание и математическое обоснование метода квантизации. Рассматривается и выводится алгоритм эффективного матричного произведения. Обосновывается формат хранения данных с точки зрения архитектуры современных процессоров. 

В практической части описывается процесс поддержки нового квантизованного слоя в рамках нейросетевого фреймворка Samsung ONE, включая сериализацию  и десериализацию параметров операции, поддержку слоя в промежуточном графовом представлении и создание вычислительного ядра в рамках интерпретатора нейросетевых моделей. Приводится анализ показателей квантизованного слоя в сравнении с исходной обученной глубокой моделью.
