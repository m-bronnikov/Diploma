\Introduction

\begin{center}
ВВЕДЕНИЕ
\end{center}

Нейронные  сети - мощный и современный инструмент для решения огромного множества задач, начиная от предсказания цен на недвижимость, основываясь на имеющихся характеристиках жилья, заканчивая генерацией или обработкой изображений в задачах компьютерного зрения. Нейронные сети сегодня используются повсюду, мы сталкиваемся с результатами их работы буквально каждый день. Однако возможность их применения ограничивается их ресурсоёмкостью: современные модели имеют миллионы настраиваемых параметров, что накладывает свой отпечаток не только на процесс обучения сети, ведь для обучения одной state-of-the-art нейронной модели необходимо иметь дорогостоящую ЭВМ или даже целый кластер, связанный в единую вычислительную систему, потратить огромное количество времени и электричества на каждую из попыток запуска алгоритма обучения при фиксированном наборе гиперпараметров, но и на использование уже обученных сетей в прикладных задачах, особенно в реальном времени, поскольку перемножение вещественных матриц, лежащее в основе работы практически любых нейронных моделей, является крайне затратной операцией, особенно, когда количество перемножаемых элементов велико.
 
Над процессом оптимизации работы нейронных сетей на сегодняшний день бьются тысячи ученых и программистов по всему миру, ведь возможность исполнять сети на недорогих устройствах, таких как микроконтроллеры и мобильные телефоны, при как можно меньших затратах ресурсов (энергии, оперативной и внешней памяти, времени на расчеты) привлекает практически любую компанию, заинтересованную в извлечении экономической прибыли на продуктах с использованием глубокого обучения. 
Особенно хорошо такая перспектива вписывается в концепцию интернета вещей (IoT), поскольку соединение нескольких <<умных>> устройств в сеть, с возможностью взаимодействовать друг с другом, передавая уже не <<сырые>>, а обработанные локально данные, может заметно упростить повседневную жизнь пользователя и улучшить материальное благосостояние того, кто сможет предложить дешевое и эффективное решение для описанной проблемы. 

Для того, чтобы снизить объем вычислений и размер занимаемой памяти, поступающие на вход глубоким моделям данные можно обрабатывать классическими математическими алгоритмами, поскольку простейшая обработка может позволить специалисту по проектированию методов машинного обучения уменьшить глубину сетей и количество нейронов в них при наименьшей потере точности модели. 
% Здесь нельзя не вспомнить о таких алгоритмах, как фильтры Гаусса и Собеля, быстрые дискретные преобразования(Фурье, вейвлет, быстрое косинусное), поскольку они хорошо математически обоснованы и позволяют быстро очистить данные от лишнего (шума) и сделать акцент сети на более важных для нее признаках в поступаемой карте. 
Но даже этого становится недостаточно на сегодняшний день, потому что растущий интерес к искусственному интеллекту побуждает придумывать и применять все новые и новые оптимизации и эвристики для повышения эффективности моделей. Среди наиболее известных следует отметить пруннинг (подрезка сети, уменьшение количества параметров), дистилляцию (создание сети меньшего размера, обученной подражать исходной) и квантизацию (переход к другому типу данных парметров сети), являющуюся на текущий момент одним из самых интересных и эффективных методов улучшения качества работы нейронной сети. 

Целью данной работы является описание и реализация метода низкобитовой квантизации предварительно обученных нейронных сетей вместе с эффективным алгоритмом работы квантизованных слоев в целочисленном представлении на современных архитектурах ЭВМ.

Для осуществления поставленной цели необходимо выполнить следующие задачи:
\begin{itemize}
    \item Изучить существующие алгоритмы квантизации и проанализировать их преимущества и недостатки.
    \item Выбрать фреймворк, в рамках которого можно реализовать предлагаемую в этой работе функциональность.
    \item Подобрать схему хранения и использования квантизованных значений и параметров.
    \item Добавить в выбранный фреймворк нейронный слой с квантизованными параметрами.
    \item Реализовать вычислительное ядро для работы слоя.
    \item Написать инструмент квантизации сетей.
    \item Проанализировать показатели полученных моделей.
\end{itemize}
