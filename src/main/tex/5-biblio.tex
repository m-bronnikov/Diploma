\Biblio
\begin{center}
СПИСОК ИСПОЛЬЗОВАННЫХ ИСТОЧНИКОВ
\end{center}


\bibliographystyle{gost780u}
\begin{thebibliography}{9}
    \bibitem{lq} Dongqing Zhang, Jiaolong Yang, Dongqiangzi Ye, Gang Hua. (2018). LQ-Nets: Learned Quantization for Highly Accurate and Compact Deep Neural Networks. Proceedings of the European Conference on Computer
Vision (ECCV), 2018, pp. 365-382

    \bibitem{quantization}
    Raghuraman Krishnamoorthi. (2018). Quantizing deep convolutional networks for efficient inference: A whitepaper. arXiv:1806.08342.

    \bibitem{bqgemm} Yongkweon Jeon, Baeseong Park, Se Jung Kwon, Byeongwook Kim, Jeongin Yun, and Dongsoo Lee. (2020). BiqGEMM: matrix multiplication with lookup table for binary-coding-based quantized DNN's. arXiv:2005.09904.
    
    \bibitem{bench} Prateeth Nayak, David Zhang, Sek Chai. (2019). Bit Efficient Quantization for Deep Neural Networks arXiv:1910.04877.
    
    \bibitem{discret} А. Ю. Морозов, Д. Л. Ревизников, К. К. Абгарян. (2019). Вопросы реализации нейросетевых алгоритмов на мемристорных кроссбарах. Известия высших учебных заведений. Материалы электронной техники. 2019. Т. 22, № 4. C. 272—278.
    
    \bibitem{conv} Shujian Yu, Kristoffer Wickstrøm, Robert Jenssen, Jose C. Principe. (2020). Understanding Convolutional Neural Networks with Information Theory: An Initial Exploration. arXiv:1804.06537
    
    \bibitem{im2col} Sangkug Lym, Donghyuk Lee, Mike O’Connor, Niladrish Chatterjee, Mattan Erez. DeLTA: GPU Performance Model for Deep Learning Applications with In-depth Memory System Traffic Analysis. arXiv:1904.01691.
    
    \bibitem{gru} Rahul Dey, Fathi M. Salem. (2017). Gate-Variants of Gated Recurrent Unit (GRU) Neural Networks. arXiv:1701.05923.
    
    \bibitem{memristor} M.S. Tarkov, M.I. Osipov. (2016). The memristor crossbar-based WTA neural network. Bull. Nov. Comp. Center, Comp. Science, 39 (2016), 69–75.
    
    \bibitem{lstm} Ralf C. Staudemeyer, Eric Rothstein Morris. (2019). Understanding LSTM: a tutorial into Long Short-Term Memory Recurrent Neural Networks. arXiv:1909.09586.
    
    \bibitem{rnn} Ф.М. Гафаров, А.Ф. Галимьянов. (2018). Искусственные нейронные сети и их приложения. Издательство Казанского университета, 2018.
    
    \bibitem{intel} Intel. (2007). Intel SSE4 Programming Reference. 2006-2007 Intel Corporation.
    
    \bibitem{neon} Andrey Kamaev. (2012). Arm NEON SIMD. URL: http://www.itlab.unn.ru/file.php?id=731
    
\end{thebibliography}